\section{Considerações finaiss}
\label{sec:consideracoes-finais}
Estudamos o desenvolvimento de software empírico e suas características, assim como a forma que esse desenvolvimento está ligado com testes automatizados e como essas práticas podem ser aplicadas em um ambiente de desenvolvimento.
%
Verificamos que é possível aplicar grande parte das práticas de BDD e TDD no desenvolvimento para a plataforma Noosfero, quando esse desenvolvimento está ainda no levantamento da história.
%
Com as pesquisas realizadas sobre usabilidade de software, podemos notar que existem estudos na área onde foram criadas metodologias que unem tanto as abordagem ágeis com a abordagem centrada no usuário. É possível fazer a integração das abordagens, mas é necessário que tenha algumas adaptações.

Assim, chegamos a algumas hipóteses que serão respondidas na segunda fase deste trabalho.

\begin{itemize}
\item Como inserir os princípios de usabilidade dentro do processo desenvolvimento empírico de software?
\item É possível alcançar melhores resultados em testes de usabilidade utilizando práticas do BDD e TDD, durante o desenvolvimento de software?
\end{itemize}

Nesse contexto, a ideia do estudo inicial sobre projeto Participa.Br foi conhecer como funciona algumas técnicas de avaliação da usabilidade.
%
Propomos algumas técnicas para análise do perfil do usuário: como aplicação de questionário de perfil de uso, análise de dados estatísticos e criação de persona do usuário.
%
Para analisar o grau de satisfação do usuário foi feito escolhido o PSSUQ por ser um questionário que possui maior grau de confiabilidade e que retorna quatro fatores, sendo esses: Satisfação Geral, Qualidade da Interface, qualidade da informação e utilidade do sistema.
%
Também foi escolhido o questionário ASQ, aplicado depois de cada tarefa executada.
%
Aplicamos as técnicas de usabilidade pesquisadas durante o trabalho, em um processo baseado em BDD e TDD, a fim de verificar problemas de usabilidade, e satisfação e uso em um estudo de caso específico, no caso plataforma Noosfero. 
%
Assim, partiremos para a segunda fase do trabalho, que será aplicar o estudo realizado, a fim de verificar a influência de testes automatizados na usabilidade do sistema. Outro passo a ser realizado é verificar os padrões de design e usabilidade adotados pelo Noosfero, propor e implementar possíveis melhorias.


%e software livre muitos outros métodos são mais viáveis ~\cite{borchardt2011}.
