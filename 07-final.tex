\section{Considerações finaiss}
\label{sec:consideracoes-finais}
Durante esta primeira parte do trabalho de conclusão de curso, estudamos o desenvolvimento de software empírico e suas características, assim como a forma que esse desenvolvimento está ligado com testes automatizados e como as práticas de testes podem ser aplicadas ou não em um ambiente real de desenvolvimento que já se encontra estabelecido.
%
Sobre o desenvolvimento de testes automatizados, verificamos que é possível aplicar grande parte das práticas de BDD e TDD no desenvolvimento de uma nova funcionalidade para a plataforma Noosfero, quando esse desenvolvimento está ainda no levantamento da história, pois observamos dificuldades de desenvolver testes quando o desenvolvimento de uma nova funcionalidade já foi iniciado, o que aconteceu no plugin para o envio de TCC, resultando num desempenho menor dos testes executados.

Com as pesquisas realizadas sobre usabilidade de software, podemos notar que existe estudos na área onde foram criadas metodologias que unem tanto as abordagem ágeis com a abordagem centrada no usuário. É possível fazer a integração das abordagens, mas é necessário que tenha algumas adaptações.

ssim, considerando o que levantamos neste trabalho, chegamos a algumas hipóteses que serão respondidas na segunda fase deste trabalho.

\begin{itemize}
\item Como inserir os princípios de usabilidade dentro do processo desenvolvimento empírico de software?
\item É possível alcançar melhores resultados em testes de usabilidade utilizando práticas do BDD e TDD, durante o desenvolvimento de software?
\end{itemize}

Nesse contexto, a ideia do estudo de caso inicial sobre projeto Participa.Br foi conhecer como funciona algumas técnicas de avaliação da usabilidade. Foi escolhido o Portal da Participação Social por ser um dos projetos apoiados pela faculdade.
%
Propomos algumas técnicas para análise do perfil do usuário: como aplicação de questionário de perfil de uso, análise de dados estatísticos e criação de persona do usuário.
%
Para analisar o grau de satisfação do usuário foi feito uma pesquisa com os principais questionários existentes e escolhemos o PSSUQ por ser um questionário que possui maior grau de confiabilidade e que retorna quatro fatores, sendo esses: Satisfação Geral, Qualidade da Interface, qualidade da informação e utilidade do sistema.
%
Também foi escolhido o questionário ASQ que é aplicado depois de cada tarefa executada. Além disso ao aplicar o teste de usabilidade é preciso observar atentamente os passos que os usuário está realizando para concluir cada tarefa.

Aplicamos as técnicas de usabilidade pesquisadas durante o trabalho, em um processo baseado em BDD e TDD, a fim de verificar problemas de usabilidade, e satisfação e uso em um estudo de caso específico, no caso plataforma Noosfero. 
%
Para concluir este estudo, finalizaremos o processo de  homologação as funcionalidades desenvolvidas (\textit{plugins}) e as mesmas serão disponibilizadas para produção.
%
Além disso, os questionários levantados (PSSUQ e ASQ) serão aplicados ao Participa.Br, para sabermos o grau de satisfação do usuário. 
%
Assim, partiremos para a segunda fase do trabalho, que será aplicar o estudo realizado  no Portal do Software Público, a fim de verificar a influência de testes automatizados na usabilidade do sistema, buscando responder as hipóteses levantadas no início deste capítulo. Outro passo a ser realizado é verificar os padrões de design e usabilidade adotados pelo Noosfero, propor e implementar possíveis melhorias.


%e software livre muitos outros métodos são mais viáveis ~\cite{borchardt2011}.