\section{Testes}
\label{sec:testes}

Para ~\cite{cotter1995} a automação de testes é uma prática ágil, eficaz e de baixo custo para melhorar
a qualidade dos sistemas de software. No entanto utilizar testes automatizados 
como uma premissa básica do desenvolvimento é um fenômeno relativamente recente, 
com início em meados  da década de 1990. Além do fato de ser uma técnica bastante utilizada pelas metodologias ágeis
de desenvolvimento.

Os testes automatizados afetam diretamente a qualidade dos sistemas de software,
portanto, agregam valor  ao produto final, mesmo que os artefatos adicionais
produzidos não sejam visíveis para os usuários finais dos sistemas~\cite{bernardo2011}.
Esses testes podem ser divididos em diversos tipos, o que facilita a manutenção 
dos mesmos:

\begin{enumerate}

\item \textbf{Testes de unidade:} testes de correção responsável por testar os 
menores trechos de código de um sistema que possui um comportamento definido e 
nomeado~\cite{bernardo2011}.
%
Normalmente, esse teste é associado a funções para linguagens procedimentais e métodos em linguagens orientadas a objetos.

\item \textbf{Testes funcionais:} testes que tem como objetivo verificar a eficiência
dos componentes de um sistema~\cite{molinari2003}.

\item \textbf{Testes de aceitação:} são testes de correção e validação, idealmente 
especificados por clientes ou usuários finais do sistema para verificar se um 
módulo funciona como foi especificado~\cite{martin2005}.
Testes de aceitação devem utilizar linguagem próxima da natural para evitar 
problemas de interpretação e de ambiguidades~\cite{cunningham2005}.

\end{enumerate}
Desenvolvimento dirigido por testes, TDD \textit{(Test-Driven Develepment)}, 
é uma técnica de desenvolvimento de software que se dá pela repetição disciplinada 
de um ciclo curto de passos de implementação de testes e do sistema~\cite{koskela2007}.
%
O ciclo curto de passos definidos por TDD cria uma dependência forte entre codificação 
e testes, o que favorece e facilita a criação de sistemas com alta testabilidade~\cite{bernardo2011}. 
%

Desenvolvimento dirigido por comportamento \textit{(BDD - Behavior Driven Development)} 
é uma prática que recomenda o mesmo ciclo de desenvolvimento de TDD, contudo, induzindo 
a utilização de uma linguagem ubíqua entre cliente e equipe de desenvolvimento, substituindo termos como assert, \textit{assert, test case, test suite} por termos 
mais comuns ao cliente, como \textit{should, context, specification}~\cite{bernardo2011}.
Embora seja principalmente uma ideia de como um processo de desenvolvimento de 
software deve ser gerenciado, a prática do BDD assume a utilização de ferramentas 
como suporte para o desenvolvimento de software~\cite{haring2011}. O BDD utiliza 
essas ferramentas para que os testes tenham como ponto de partida o comportamento 
das funcionalidade do sistema.


Testes automatizados devem ser desenvolvidos com prioridade, buscando um rápido 
\textit{feedback}, contribuindo assim com a melhoria do sistema. Para isso é 
necessário que os cenários de testes estejam bem definidos junto à equipe.
