\section{Testes}
\label{sec:testes}

A automação de testes é uma prática ágil, eficaz e de baixo custo para melhorar
a qualidade dos sistemas de software.
%
Os testes automatizados afetam diretamente a qualidade dos sistemas de software,
 mesmo que os artefatos adicionais produzidos não sejam visíveis para os usuários finais dos sistemas~\cite{bernardo2011}.
Esses testes podem ser divididos em diversos tipos, sendo os abordados neste estudo:

\begin{enumerate}

\item \textbf{Testes de unidade:} testes de correção responsável por testar os 
menores trechos de código de um sistema que possui um comportamento definido e 
nomeado~\cite{bernardo2011}.

\item \textbf{Testes funcionais:} testes que tem como objetivo verificar a eficiência
dos componentes de um sistema~\cite{molinari2003}.

\item \textbf{Testes de aceitação:} são testes de correção e validação, idealmente 
especificados por clientes ou usuários finais do sistema para verificar se um 
módulo funciona como foi especificado~\cite{martin2005}.

\end{enumerate}
Duas técnicas de desenvolvimento foram utilizadas neste estudo, o desenvolvimento dirigido por testes, TDD \textit{(Test-Driven Develepment)}, é uma técnica se dá pela repetição de um ciclo de passos de implementação de testes e do sistema~\cite{koskela2007}.
%
O que cria uma dependência forte entre codificação e testes, faciltando a a criação de sistemas com alta testabilidade~\cite{bernardo2011}. 
%
Outra técnica é o desenvolvimento dirigido por comportamento \textit{(BDD - Behavior Driven Development)} 
 que recomenda o mesmo ciclo de desenvolvimento de TDD, contudo, induzindo 
a utilização de uma linguagem ubíqua entre cliente e equipe de desenvolvimento. 
%
Embora seja principalmente uma ideia de como um processo de desenvolvimento de 
software deve ser gerenciado, a prática do BDD assume a utilização de ferramentas 
como suporte para o desenvolvimento de software~\cite{haring2011}. O BDD utiliza 
essas ferramentas para que os testes tenham como ponto de partida o comportamento 
das funcionalidade do sistema.

