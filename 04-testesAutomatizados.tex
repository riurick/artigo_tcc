\section{Testes}
\label{sec:testes}

A automação de testes é uma prática ágil, eficaz e de baixo custo para melhorar
a qualidade do software.
%
Os testes automatizados afetam diretamente a qualidade do software, mesmo que os artefatos  produzidos não sejam visíveis para os usuários finais~\cite{bernardo2011}.
Esses testes podem ser divididos em diversos tipos, sendo os abordados neste estudo:

\begin{enumerate}

\item \textbf{Testes de unidade:} testes de correção responsável pelos
menores trechos de código com um comportamento~\cite{bernardo2011}.

\item \textbf{Testes funcionais:} testes que tem como objetivo verificar a eficiência
dos componentes de um sistema~\cite{molinari2003}.

\item \textbf{Testes de aceitação:} testes para verificar se um módulo se comporta como foi especificado~\cite{martin2005}.

\end{enumerate}

Foram utilizados neste estudo, o desenvolvimento dirigido por testes, TDD \textit{(Test-Driven Develepment)}, é uma técnica se dá pela repetição de um ciclo de passos de implementação de testes e do sistema~\cite{koskela2007}.
%
E a técnica de desenvolvimento dirigido por comportamento \textit{(BDD - Behavior Driven Development)} que induz a utilização de uma linguagem ubíqua entre cliente e equipe de desenvolvimento. 

