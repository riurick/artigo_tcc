\section{Métodos Empíricos de Desenvolvimento de Software}
\label{sec:desenvolvimento-empirico}

O Empirismo baseia-se na aquisição de sabedoria através da percepção do mundo 
externo, ou então do exame da atividade da nossa mente, que abstrai a realidade 
que nos é exterior e as modifica internamente~\cite{chaui2003}.	

Software livre é uma filosofia que trata programas de computadores como fontes de 
conhecimento que devem ser compartilhados entre a comunidade.
%
De acordo com ~\cite{stallman2001}, ativista fundador do movimento software livre, um software deve seguir quatro princípios:
%
\begin{enumerate}
\item Liberdade de execução para qualquer uso;
\item Liberdade de estudar o funcionamento de um software e de adaptá-lo às suas 
necessidades
\item Liberdade de redistribuir cópias;
\item Liberdade de melhorar o software e de tornar as modificações públicas de modo 
que a comunidade se beneficie da melhoria.
\end{enumerate}

A utilização de métodos ágeis no desenvolvimento de software tem como características 
intrínsecas a flexibilidade e rapidez nas respostas a mudanças. 
%
A agilidade, para uma organização de desenvolvimento de software, é a habilidade 
de adotar e reagir rapidamente e apropriadamente a mudanças no seu ambiente e por 
exigências impostas pelos ``clientes''~\cite{nerur2005}.

Existe uma relação entre as práticas ágeis e o desenvolvimento baseado em software livre. ~\cite{corbucci2011} fala que o desenvolvimento baseado em software livre vem crescendo, porém a essência da comunidade ao redor do programa é de manter indivíduos que interajam de forma a produzir o que é mais importante. As ferramentas apenas possibilitam isso. 

O desenvolvimento baseado em software livre e as práticas ágeis compartilham os mesmos valores, pois ambos são métodos empíricos, que buscam tomar decisões rápidas de acordo com a percepção do mundo externo.

