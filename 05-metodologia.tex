\section{Usabilidade}
\label{sec:usabilidade}

Ainda, usabilidade é definida como o fator que assegura que um produto ou serviço é fácil de usar, eficiente e agradável a partir do ponto de vista do usuário~\cite{preece2007}.
Adicionalmente, existem metas de usabilidade que servem para guiar o desenvolvimento de produtos fáceis de usar, eficientes e agradáveis~\cite{preece2007}. São elas:

\begin{enumerate}
	\item \textbf{Utilidade:} é a medida na qual o sistema propicia o tipo certo de funcionalidade, de maneira que os usuários possam realizar aquilo de que precisam ou desejam.
	\item \textbf{Eficácia:} refere-se ao quanto um sistema é bom em fazer o que se espera dele.
	\item \textbf{Eficiência:} refere-se à maneira como o sistema auxilia os usuários na realização de suas tarefas.
	\item \textbf{Segurança:} implica em proteger o usuário de condições perigosas e situações indesejáveis.
	\item \textbf{Facilidade de aprendizado:} refere-se a qual fácil é aprender a usar o sistema.
	\item \textbf{Facilidade de lembrar como se usa:} refere-se à facilidade de lembrar como utiliza um sistema, depois de já ter aprendido como usar.
\end{enumerate}

Para entender o que é usabilidade e como ela está inserida no ciclo de vida do desenvolvimento de software, precisamos compreender as relações que o termo tem com as diversas áreas que a envolve: 

\emph{Design} centrado no usuário é um campo de estudo que reúne metodologias de \emph{design} nas quais o público álvo de um produto ou serviço influencia as diretrizes e requisitos do sistema~\cite{norman2006design}.
%
Para~\cite{norman2006design}, cunhador do termo, o \emph{design} centrado no usuário é uma filosofia baseada nas necessidades e interesses dos usuário, com ênfase em fazer produtos usáveis e inteligíveis.

Toda a interação com um produto, serviço ou marca. O termo é usado frequentemente para sintetizar toda a experiência com um produto de software. Não engloba  somente as funcionalidades e sim o quanto o aplicativo é  agradável ao usuário~\cite{travis2013}.
%	
O termo User Experience (UX) foi cunhado por Donald Norman na década de 80 para cobrir todos os aspectos da experiência da pessoa com o sistema. Acreditava que as definições de interface do usuário e usabilidade limitava o entendimento sobre o que o trabalho dele representava.

Um dos grandes problemas encontrados em softwares livres é a pouca atenção dada aos aspectos referentes a usabilidade e acessibilidade das aplicações. Acredita-se que um dos principais problemas que contribui para a falta de usabilidade de software livre é sua própria comunidade que apenas enfatiza na criação e, melhoria e teste do código fonte.  

Segundo ~\cite{preece2007}, uma das causas da baixa adoção de softwares livres em mercados de larga escala é a baixa qualidade dos estilos de interação implementados nas interfaces dos produtos. Uma grande maioria não se preocupa com bons elementos de interface com usuário (UI).

Entender o perfil do usuário é um dos principais pontos que devem ser levados em consideração pelos desenvolvedores de software em geral. Cada perfil de usuário tem suas particularidades e suas expectativas quanto a utilização do sistema. Quando falamos de usuários com experiência de uso de softwares semelhantes é preciso ter uma maior atenção na usabilidade para que possa ter uma boa aceitação e menor impacto com as mudanças.

Os métodos ágeis é uma abordagem de desenvolvimento que têm como princípios um conjunto de valores com foco no cliente e nas pessoas envolvidas, na entrega constante de software funcional, no desenvolvimento iterativo baseado em ciclos curtos de entrega e na aceitação da constante mudança dos requisitos ao longo do desenvolvimento. 

	Tanto os métodos ágeis e o IHC tem esses mesmos princípios, o que seria interessante a aplicação das técnicas utilizadas em IHC no contexto de uma abordagem ágil. Essa semelhança entre os valores proporciona um quadro favorável à integração minimalista de pŕaticas de IHC em ambientes ágeis. Podemos citar alguns desse valores como por exemplo: (i) ciclos curtos com entregas contínuas e incrementais, que favorecem a aplicação de técnicas de prototipagem; (ii) forte envolvimento do usuário que favorece a aplicação de princípios de projetos participativos e (iii) programação em pares onde em IHC geralmente a avaliação de usabilidade é feita em pares ~\cite{barbosa2008estrategia}. 

A integração entre os processos de usabilidade e métodos ágeis é esperada e possível visto que tanto os métodos ágeis como os processo de usabilidade em em comum características que colocam o foco do desenvolvimento nas necessidade e anseios dos usuários finais, na interação entre os \textit{stakeholders} envolvidos e na qualidade final do produto a ser desenvolvido.