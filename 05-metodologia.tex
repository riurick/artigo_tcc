\section{Usabilidade}
\label{sec:usabilidade}

A usabilidade é definida como o fator que assegura que um produto é fácil de usar, eficiente e agradável~\cite{preece2007}.
Adicionalmente, existem metas de usabilidade que servem para guiar o desenvolvimento~\cite{preece2007}:

\begin{enumerate}
	\item \textbf{Utilidade:} é a medida na qual o sistema propicia o tipo certo de funcionalidade, de maneira que os usuários possam realizar aquilo de que precisam ou desejam.
	\item \textbf{Eficácia:} refere-se ao quanto um sistema é bom em fazer o que se espera dele.
	\item \textbf{Eficiência:} refere-se à maneira como o sistema auxilia os usuários na realização de suas tarefas.
	\item \textbf{Segurança:} implica em proteger o usuário de condições perigosas e situações indesejáveis.
	\item \textbf{Facilidade de aprendizado:} refere-se a qual fácil é aprender a usar o sistema.
	\item \textbf{Facilidade de lembrar como se usa:} refere-se à facilidade de lembrar como utiliza um sistema, depois de já ter aprendido como usar.
\end{enumerate}

Para entender o que é usabilidade e como ela está inserida no ciclo de vida do desenvolvimento de software, precisamos compreender as relações que o termo tem com as diversas áreas que a envolve: 


Para~\cite{norman2006design}, \emph{design} centrado no usuário é uma filosofia baseada nas necessidades e interesses dos usuário, com ênfase em fazer produtos usáveis e inteligíveis.
%
O termo é usado frequentemente para sintetizar toda a experiência com um produto de software. Não engloba  somente as funcionalidades e sim o quanto é agradável ao usuário~\cite{travis2013}.
%	
O termo User Experience (UX) pode ser definida como a experiência da pessoa com o sistema~\cite{norman2006design}.
%
Um problea encontrad em softwares livres é a pouca atenção aos aspectos de usabilidade. Acredita-se que um dos principais problemas que contribui para a falta de usabilidade de software livre é sua própria comunidade que apenas enfatiza na criação e, melhoria e teste do código fonte.  
%
Entender o perfil do usuário é um dos principais pontos que devem ser levados em consideração pelos desenvolvedores de software. Cada perfil tem particularidades e expectativas quanto a utilização do sistema. 

Os métodos ágeis, como abordagem de desenvolvimento, têm seus valores no foco nas pessoas envolvidas, na entrega constante, no desenvolvimento iterativo e na aceitação da mudança dos requisitos.. 
%
A integração entre os processos de usabilidade e métodos ágeis é esperada e possível visto que tanto os métodos ágeis como os processo de usabilidade tem em comum características que colocam o foco do desenvolvimento nas necessidade dos usuários finais.
