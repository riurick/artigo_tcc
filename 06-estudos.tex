\section{Estudo}
\label{sec:estudo}

Foi planejado um estudo de usabilidade, cujo o objetivo é analisar a interação dos usuários com o portal Participa.Br a fim de avaliar a qualidade em uso do portal. 
%
Assim, foram definidas questões sobre o que é preciso saber de forma a apoiá-la a entender se o objetivo foi alcançado, e para cada questão foram definidas as métricas relacionadas na Tabela~\ref{tabela-questoes}: 

\begin{table}[h]
\scalefont{.85}
\begin{tabular}{|l|l|l|}
\hline
\textbf{Questões}  & Métricas    & \begin{tabular}[c]{@{}l@{}}Diretrizes para \\ interpretação \end{tabular}  \\ \hline
\begin{tabular}[c]{@{}l@{}}Q1. Qual o perfil do\\usuário que utiliza\\o  Participa.Br?\end{tabular} &    Perfil    & \begin{tabular}[c]{@{}l@{}} Análise de Dados Estatísticos,\\ criação de personas, análise \\dos dados qualitativos\end{tabular}\\ \hline
\begin{tabular}[c]{@{	}l@{}}Q2. Qual o grau de \\ satisfação do usuário\\do Participa.Br\end{tabular} &  \begin{tabular}[c]{@{}l@{}}Satisfação\\ do usuário\end{tabular} & \begin{tabular}[c]{@{}l@{}}Escore da satisfação global\\pelo usuário\end{tabular}                               \\ \hline
\begin{tabular}[c]{@{}l@{}} Q3. Quantidade de tempo\\gasto para realizar\\as tarefas\end{tabular}&      Duração        & Tempo gasto             \\ \hline
\end{tabular}
\caption{Questões de Pesquisa}
\scalefont{1}
\label{tabela-questoes}
\end{table}

Elencamos algumas técnicas para identificação dos usuários:

\begin{enumerate}
\item \textbf{Dados Estatísticos} : Possibilita identificar algumas informações sobre o perfil dos usuários, coletadas de base de dados, redes sociais, etc.

\item \textbf{Questionário de identificação de perfil dos usuários:} Pesquisa que busca compreender quem são, qual o conhecimento e como utilizam o sistema. 

\item \textbf{Identificação de Personas:} “Persona” são personagens fictícios criados com base em dados reais.  
\end{enumerate}

Elencamos algumas técnicas para avaliar a usabilidade do portal Participa.Br:

\begin{table}[h]
\scalefont{.85}
\begin{tabular}{|l| p{5cm} |}
\hline
Técnica & Descrição \\ \hline
Observar Usuários & Um observador irá registrar o tempo 
gasto por cada participante para concluir o estudo de caso, 
avaliar a ferramenta e se necessitou de alguma ajuda    \\ \hline
Perguntar aos usuários & Os questionários ASQ e PSSUQ 
de satisfação dos usuários será utilizado 
para coletar as opiniões dos participantes.\\ \hline
\end{tabular}
\caption{Técnicas de avaliação para os testes com usuários}
\scalefont{1}
\label{tabela-tecnicas}
\end{table}

Os instrumentos de coletas de informações utilizados são dois questionários utilizados para medir a satisfação do usuário.
%
São eles o \textit{After-Scenario Questionnaire} (ASQ) \footnote{ASQ: Proposto por Lewis}, destinado ao uso em testes de usabilidade baseados em cenários. Possui três itens: (1) facilidade de conclusão da tarefa, (2) tempo necessário para completar uma tarefa e,(3) a adequação das instruções ou materiais de apoio fornecidos. Ainda temos o \textit{Post-Study System Usabiliy Questionnaire} (PSSUQ), aplicado após a conclusão de todos os cenários para fornecer uma avaliação  da usabilidade do sistema de forma mais ampla, podendo avaliar 4 fatores (satisfação geral, utilidade do sistema, qualidade da interface e qualidade da informação). 

%+++++++++++++++++++++++++++++++++++++++++++++++++++++++++++++++++++++++++++++++++++++++++++%

O estudo sobre testes teve seu enfoque na rede Comunidade UnB, sendo desenvolvidos alguns \textit{plugins}. A rede Comunidade UnB 
necessita de restrição de acesso aos usuários, para que somente membros ativos da universidade tenham acesso. 
%
Assim foi desenvolvido um \textit{plugin} no noosfero, que efetuasse as restrições necessárias, utilizando o protocolo de autenticação da UnB, o LDAP (\textit{Lightweight Directory Access Protocol}).

Com os testes unitário e funcionais do plugin alcançamos os seguintes dados:

\begin{itemize}
\item Quantidade de testes executados: \textbf{96 testes;}
\item Quantiadde de falhas obtidas: \textbf{0 falhas;}
\item Taxa de cobertura de código: \textbf{88.94;}
\end{itemize}

Outro \textit{plugin} (plugin de Envio de TCC) desenvolvido para o Noosfero, porém em uma aplicação diferente, o Portal UnB Gama, é responsável por criar uma atribuição de trabalhos, chamada de \textit{work assignment}. Essa atribuição possui algumas funcionalidades específicas como possibilitar que os usuários envolvidos sejam notificados via \textit{email} sobre a submissão de um certo trabalho. Segue os dados sobre a execução dos testes:

\begin{itemize}
\item Quantidade de testes executados: \textbf{28 testes};
\item Quantiadde de falhas obtidas: \textbf{0 falhas};
\item Taxa de cobertura de código: \textbf{73.30;}
\end{itemize}

Para o plugin de envio de tcc, foram definidos testes de aceitação, para verificar o comportamento da funcionalidade:

\begin{itemize}
\item Quantidade de cenários executados: \textbf{6 cenários};
\item Quantidade de passos executadas: \textbf{130 passos};
\item Quantiadde de falhas obtidas: \textbf{0 falhas};
\end{itemize}

